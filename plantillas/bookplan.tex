\documentclass[letterpaper,12pt]{book} % Formato

% Paquetes
\usepackage[utf8]{inputenc}
\usepackage[spanish]{babel}
\usepackage[margin=2.5cm, top=2.5cm, includefoot]{geometry}
\usepackage{graphicx} % Insertar imágenes
\usepackage{fancyhdr}
\usepackage[hidelinks]{hyperref} % Gestión de hipervínculos
\usepackage{parskip} % No sangría
\usepackage{anyfontsize}
\usepackage{romande}
\usepackage[T1]{fontenc}
\usepackage{setspace}
\usepackage[
    backend=biber,
    style=apa,
  ]{biblatex}
\usepackage{csquotes}
\DeclareLanguageMapping{spanish}{spanish-apa}
\addbibresource{~/bib/refs.bib} % Cambia el directorio de tu archivo .bib que contiene tu bibliografía

% Variables - Cambia el directorio de imagen a usar
\newcommand{\logouacm}{/home/frvnzj/Escritorio/LaTeX/Images/uacm.jpg}

\newcommand{\startdate}{Febrero 05 del 2023}

\renewcommand{\thechapter}{\Roman{chapter}}
\fancyhf{}
\spacing{1.5}

% -------------DOCUMENTO-----------------

\begin{document}

  % -------------PORTADA-------------------
    \begin{titlepage}

      \centering

      \Large\underline{NOMBRE DEL AUTOR}
      
      \par\vspace{2cm}
      
      \fontsize{40}{20}\selectfont TÍTULO DE TESIS
      \par

      \vfill

      \includegraphics[width=0.3\textwidth]{\logouacm}

    % Si requieres fecha, descomenta las siguientes dos líneas
      % \vfill
      % \large\startdate

    \end{titlepage}

    \clearpage
    \thispagestyle{empty}


% ------------ ÍNDICE------------
\tableofcontents 


% -------------ESCRITO-------------
   \clearpage
   \thispagestyle{empty}

  % ------------INTRODUCCIÓN----------

    \cleardoublepage\phantomsection\addcontentsline{toc}{chapter}{Introducción}
    
    
    \pagestyle{plain}    
    \chapter*{Introducción}

Lorem ipsum dolor sit amet, qui minim labore adipisicing minim sint cillum sint consectetur cupidatat.
    
\par

\clearpage

  % -------------CAPÍTULO 1----------------

\chapter{Nombre del capítulo 1} 

\thispagestyle{empty}
\pagestyle{empty}

\cleardoublepage

\pagestyle{headings}

Lorem ipsum dolor sit amet, officia excepteur ex fugiat reprehenderit enim labore culpa sint ad nisi Lorem pariatur mollit ex esse exercitation amet. Nisi anim cupidatat excepteur officia. Reprehenderit nostrud nostrud ipsum Lorem est aliquip amet voluptate voluptate dolor minim nulla est proident. Nostrud officia pariatur ut officia. Sit irure elit esse ea nulla sunt ex occaecat reprehenderit commodo officia dolor Lorem duis laboris cupidatat officia voluptate. Culpa proident adipisicing id nulla nisi laboris ex in Lorem sunt duis officia eiusmod. Aliqua reprehenderit commodo ex non excepteur duis sunt velit enim. Voluptate laboris sint cupidatat ullamco ut ea consectetur et est culpa et culpa duis.

Este es un ejemplo de cómo usar las citas
\parencite{Sartlp}, si quieres añadir páginas este
es el comando \parencite[50-60]{parmh}. Y este es
un ejemplo de notas al pie de
página\footnote{Lorem ipsum dolor sit amet, qui
minim labore adipisicing minim sint cillum sint
consectetur cupidatat.}.


\section{Sección}

Lorem ipsum dolor sit amet, qui minim labore adipisicing minim sint cillum sint consectetur cupidatat.

% -------CAPÍTULO 2----------------

    \clearpage
\thispagestyle{empty}

    \chapter{Nombre del capítulo 2}
\thispagestyle{empty}
\pagestyle{empty}

\cleardoublepage

\pagestyle{headings}

Lorem ipsum dolor sit amet, qui minim labore adipisicing minim sint cillum sint consectetur cupidatat.

\section{Sección}

Lorem ipsum dolor sit amet, qui minim labore adipisicing minim sint cillum sint consectetur cupidatat.

% --------bibliografía

    \clearpage
\thispagestyle{empty}
\cleardoublepage

\pagestyle{headings}

% Bibliografía

    \cleardoublepage\phantomsection\addcontentsline{toc}{chapter}{Bibliografía}
    
    \nocite{*} % Comando que “imprime” la bibliografía no referenciada
    \printbibliography % Comando que “imprime la bibliogrfía referenciada


\end{document}
