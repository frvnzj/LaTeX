\documentclass[doc,letterpaper,12pt]{apa7}

\usepackage[utf8]{inputenc}
\usepackage[greek,spanish]{babel}
\usepackage{hyperref}
\usepackage{graphicx}
\usepackage[style=apa,backend=biber]{biblatex}
\usepackage{csquotes}
\DeclareLanguageMapping{spanish}{spanish-apa}
\addbibresource{~/bib/refs.bib}

\usepackage[T1]{fontenc}
\usepackage{baskervillef}
\usepackage[varqu,varl,var0]{inconsolata}
\usepackage[scale=.95,type1]{cabin}
\usepackage[baskerville,vvarbb]{newtxmath}
\usepackage[cal=boondoxo]{mathalfa}

\usepackage{yfonts}

\usepackage{fancyhdr}
\pagestyle{fancy}
\renewcommand{\headrulewidth}{0.4pt}
\fancyhead{}
\fancyhead[C]{\emph{título}}
\fancyhead[LO,RE]{nombre/other}
\fancyhead[LE,RO]{\thepage}

%--------------------------------------------------------

\begin{document}

\title{\Huge{Título}}

\author{Nombre del Autor}
\affiliation{}

\maketitle

\section{Introducción}

Lorem ipsum dolor sit amet, qui minim labore adipisicing minim sint cillum sint consectetur cupidatat.

Este es un ejemplo de cómo usar las citas
\parencite{Sartlp}, si quieres añadir páginas este
es el comando \parencite[50-60]{parmh}. Y este es
un ejemplo de notas al pie de
página\footnote{Lorem ipsum dolor sit amet, qui
minim labore adipisicing minim sint cillum sint
consectetur cupidatat.}.

% ---------------Bibliografía

\nocite{*} % Comando que “imprime” la bibliografía no referenciada
\printbibliography % Comando que “imprime la bibliogrfía referenciada


\end{document}
